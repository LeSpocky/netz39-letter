%%
%% This is file `beispiel.tex',
%% generated with the docstrip utility.
%%
%% The original source files were:
%%
%% g-brief.dtx  (with options: `beispiel')
%% 
%% =======================================================================
%% 
%% Copyright (C) 1991-2003 Michael Lenzen.
%% 
%% For additional copyright information see further down in this file.
%% 
%% This file is part of the G-BRIEF package.
%% -----------------------------------------------------------------------
%% 
%% It may be distributed under the terms of the LaTeX Project Public
%% License LPPL), as described in lppl.txt in the base LaTeX distribution.
%% Either version 1.1 or, at your option, any later version.
%% 
%% The latest version of this license is in
%% 
%%          http://www.latex-project.org/lppl.txt
%% 
%% LPPL Version 1.1 or later is part of all distributions of LaTeX
%% version 1999/06/01 or later.
%% 
%% 
%% Error reports in case of UNCHANGED versions to
%% 
%%                            <lenzen@lenzen.com>
%%                            <m.lenzen@t-online.de>
%% 
%% 
\def\filedate{2008/07/15}
\def\fileversion{4.0.2}

\documentclass[12pt,ngerman,latin9,a4paper]{g-brief-ntz39lg19_exp}

\lochermarke
\faltmarken
\fenstermarken
\unserzeichen
 \trennlinien
%% \klassisch

\Name                {Netz39 e.\,V.}
\Strasse             {Leibnizstra\ss e 32}
\Zusatz              {}
\RetourAdresse       {}
\Ort                 {39104 Magdeburg}
\Land                {}

\Telefon             {+49\ 391\ 58245281}
\Telefax             {}
\Telex               {}
\HTTP                {http://www.netz39.de}
\EMail               {kontakt@netz39.de}

\Vorsitz             {Stefan Haun}
\Vertreter           {Frank Blaschke}
\Kassenwart          {David Kilias}

\Bank                {Volksbank Magdeburg eG}
\BLZ                 {810\ 932\ 74}
\Konto               {144\ 248\ 1}

\Registergericht     {Amtsgericht Stendal}
\Registernummer      {VR\ 3169}

\Umsatzsteuerid      {}
\Steuernummer        {}

\Unterschrift        {Manni Mitglied}

\Postvermerk         {}      % E I N S C H R E I B E N}
\Adresse             {Frau und Herr\\
                      Willi Geldgeber\\
                      Fass Ohne Boden 1\\
                      \\
                      D-12345 Goldstadt
                      }

\Betreff             {Foo f\"ur die Foobar}

\Datum               {\today}
\IhrZeichen          {}
\IhrSchreiben        {}
\MeinZeichen         {Netz39}

\Anrede              {Sehr geehrte willige Geldgeber,}
\Gruss               {Mit freundlichen Gr\"u\ss{}en}{1cm}

\Anlagen             {}
\Verteiler           {}

\begin{document}
\begin{g-brief}

mit Bedauern m\"ussen wir Ihnen mitteilen, dass uns schon wieder das Foo f\"ur die Bar ausgegangen ist. Und obwohl wir durch unlautere Mittel versucht haben, die Mehreinnahmen aus der Mate-Kasse abzusch\"opfen, reicht dies bei Weitem nicht aus, um die Kosten f\"ur die Foobar zu decken.

Wir freuen uns Ihnen daher mitteilen zu k\"onnen, dass Sie uns ohne Bedingungen sofort einen 5 (in Worten: f\"unf)-stelligen Betrag \"uberweisen d\"urfen.

Ein kurzes Zitat als Motivation:

\begin{quote}
One has to imagine, as well as one may, the fate of those batteries towards Esher, waiting so tensely in the twilight.  Survivors there were none.  One may picture the orderly expectation, the officers alert and watchful, the gunners ready, the ammunition piled to hand, the limber gunners with their horses and waggons, the groups of civilian spectators standing as near as they were permitted, the evening stillness, the ambulances and hospital tents with the burned and wounded from Weybridge; then the dull resonance of the shots the Martians fired, and the clumsy projectile whirling over the trees and houses and smashing amid the neighbouring fields.

One may picture, too, the sudden shifting of the attention, the swiftly spreading coils and bellyings of that blackness advancing headlong, towering heavenward, turning the twilight to a palpable darkness, a strange and horrible antagonist of vapour striding upon its victims, men and horses near it seen dimly, running, shrieking, falling headlong, shouts of dismay, the guns suddenly abandoned, men choking and writhing on the ground, and the swift broadening-out of the opaque cone of smoke.  And then night and extinction--nothing but a silent mass of impenetrable vapour hiding its dead.

Before dawn the black vapour was pouring through the streets of Richmond, and the disintegrating organism of government was, with a last expiring effort, rousing the population of London to the necessity of flight.

So you understand the roaring wave of fear that swept through the greatest city in the world just as Monday was dawning--the stream of flight rising swiftly to a torrent, lashing in a foaming tumult round the railway stations, banked up into a horrible struggle about the shipping in the Thames, and hurrying by every available channel northward and eastward.  By ten o'clock the police organisation, and by midday even the railway organisations, were losing coherency, losing shape and efficiency, guttering, softening, running at last in that swift liquefaction of the social body.

All the railway lines north of the Thames and the South-Eastern people at Cannon Street had been warned by midnight on Sunday, and trains were being filled.  People were fighting savagely for standing-room in the carriages even at two o'clock.  By three, people were being trampled and crushed even in Bishopsgate Street, a couple of hundred yards or more from Liverpool Street station; revolvers were fired, people stabbed, and the policemen who had been sent to direct the traffic, exhausted and infuriated, were breaking the heads of the people they were called out to protect.

And as the day advanced and the engine drivers and stokers refused to return to London, the pressure of the flight drove the people in an ever-thickening multitude away from the stations and along the northward-running roads.  By midday a Martian had been seen at Barnes, and a cloud of slowly sinking black vapour drove along the Thames and across the flats of Lambeth, cutting off all escape over the bridges in its sluggish advance.  Another bank drove over Ealing, and surrounded a little island of survivors on Castle Hill, alive, but unable to escape.

So he got out of the fury of the panic, and, skirting the Edgware Road, reached Edgware about seven, fasting and wearied, but well ahead of the crowd.  Along the road people were standing in the roadway, curious, wondering.  He was passed by a number of cyclists, some horsemen, and two motor cars.  A mile from Edgware the rim of the wheel broke, and the machine became unridable.  He left it by the roadside and trudged through the village.  There were shops half opened in the main street of the place, and people crowded on the pavement and in the doorways and windows, staring astonished at this extraordinary procession of fugitives that was beginning.  He succeeded in getting some food at an inn.
\end{quote}

Hier endet das Zitat. Na? Haben Sie es erkannt?

Wir danken im Voraus.

\end{g-brief}
\end{document}


\endinput
%%
%% End of file `beispiel.tex'.
